\message{ !name(rchain-sites.tex)}\documentclass[12pt, letterpaper]{article}
\usepackage{blindtext}
\usepackage[T1]{fontenc}
\usepackage[utf8]{inputenc}
 
\title{CI/CD static websites}
\author{Kayvan Kazeminejad}
\date{\today}
 
\begin{document}

\message{ !name(rchain-sites.tex) !offset(-3) }

 
\maketitle
 
\section{Introduction}
 
I recently was tasked to do a migration of all 6 RChain web sites. Theses
sites are three node-js apps with poestgres backend,  a multi-home wordpress, and two static html web sites. The ideal solution would:
\item scalabel, scale up/down to meet demands
\item extendable, easily add additional websites/webapps
\item utilize href\{https://en.wikipedia.org/wiki/CI/CD}{CI/CD} pipeline with no manual intervention
\item provide traceability 
\item provide fast rapid deployments with no downtime
\item href\{https://en.wikipedia.org/wiki/Transport_Layer_Security}{SSL/TLS certificates}

Luckly, we had a solution in \href{https://github.com/rchain/rsong-proxy}
{RSong} that met all above non-functional requirements. As an \href{https://en.wikipedia.org/wiki/Infrastructure_as_code}{IaC}, RSong
provides all foregoing features.  
\item CI/CD by utilizing google href\{https://cloud.google.com/cloud-build/}{cloud-build} and git 
\item scalable using href\{https://kubernetes.io}{kubernetes, k8} clustering technology.
\item SSL/TLS certificates by utilizing k8 ingress and href\{https://letsencrypt.org}{letsencrypt}

\section{The Pipeline} 
We created a very simple work flow.  Authors or developers will modify the site
\item create a git repository
\item create a build script to package all assets
\item create docker image

  \section{Step 1: git}
  The first step was to set up a git repo.  
  \section{Step 2: CI/CD}
  CI/CD must not involve manual work code is committed. Consequently for each
  website/webapp we set up a gulp task to package the source code for docker packaging.
  \section{Step 3: docker build and publish}
  Once the code was packaged, the next step was to move all the code to Dockerfile
  \section{Step 4: apply}
  
For all websites/webapps, we create
\item git repository
\item template build script in GULP
\item href\{https://github.com/rchain/www/blob/master/rsong/Dockerfile}{Dockerfile}
  
  
\message{ !name(rchain-sites.tex) !offset(-56) }
